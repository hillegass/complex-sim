\documentclass{article}
\usepackage[utf8]{inputenc}
\usepackage{url}
\usepackage[margin=0.75in]{geometry}

\setlength{\parskip}{0.7em}
\setlength{\parindent}{0em}

\begin{document}
	\begin{center}
		
		% MAKE SURE YOU TAKE OUT THE SQUARE BRACKETS
		\LARGE{\textbf{CSE 6730, Group 37 Proposal}} \\
		\vspace{1em}
		\Large{Project 2: Complex Simulation} \\
		
	\end{center}
	\begin{normalsize}
		
		\section{Project Title}
		
		Simulation of Predator-Prey Population Dynamics
		
		\section{Team Members}
		
		\begin{enumerate}
			\item D. Aaron Hillegass (GTID 901988533)
			\item Siawpeng Er (GTID 903413430)
			\item Xiaotong Mu (GTID 903529807)
		\end{enumerate}
		
		\section{Problem Description and Purpose}
		The predator and prey relationship is an important ecological system. Their populations rise and fall over time as they interact and impact one another. These interactions are the prime movers of energy through food chains. One of the mathematical models that simulates predator and prey interactions is the Lotka-Volterra model proposed by Alfred Lotka and Vito Volterra. Lotka helped develop the logistic equation to explain autocatalytic chemical reactions. Volterra interconnected the logistic equation to two separate populations in competition to explain predator and prey relationships. The basics of the Lotka-Volterra model can be used to describe many fundamental characteristics of ecological systems and can even be extended to other ideas like military response.
		
		
		\section{Data Source}
		-pending
		
		\section{Methodology}
		Our simulation will first simulate predators and prey entering and exiting a predefined area. Both prey and predators are affecting each other. In simplest interaction, predators depend on the prey as the food source. However, any abuse of the food source may result in decease in population of the prey, and subsequently decrease the number of the predators due to lack of food. Because of such interaction, the population of the predators and the prey may oscillate, and inversely proportional to each others. 
		
		Traditionally, there is the nonlinear Lotka-Volterra Model of the predator-prey dynamic system \cite{inproceedings, 1102729}. LVM approach is a simplified model and suitable for detailed stability analysis. However, it is also very limited model and lack of flexibility for complex interaction. Hence, we also hope to incorporate the Agent-Based Model \cite{Hodzic} in this project to increase the completeness of our analysis. 
		
		In our project, some of the ideas that we wish to investigate include:
		\begin{enumerate}
			\item Long-term population interaction among predators and prey.
			\item Introduction of the uncertainties like diseases.
			\item Introduction of the third parties interaction: human activity, natural disasters etc.
		\end{enumerate}
		
		\section{Development Platform}
		The programming language is Python 3. We will provide a Jupyter notebook for user interaction.
		In the Jupyter notebook, we will allow the user to change some of the probability and the simulation parameters to see different result of the simulation.
		
		\section{Division of Labor}
		As we move forward on our project, we plan to work concurrently. The timeline is as below:
		
		\begin{center}
			\begin{tabular}{ |c|c|c| } 
				\hline
				Task & Duration  \\ 
				\hline
				Data collection & 2 weeks \\ 
				Modeling design and implementaion & 4 weeks \\ 
				Modeling revised & 4 weeks \\ 
				\hline
			\end{tabular}
		\end{center}
		

		
		\bibliographystyle{plain}
		\bibliography{reference}
	\end{normalsize}
	
\end{document}
